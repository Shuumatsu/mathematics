\documentclass{article}
\usepackage{amsmath}

\begin{document}

\begin{enumerate}
	\item{
	      \textbf{Axiom.}
	      \emph{If A is a set then A is also an object.
		      In particular, given two sets A and B, it is meaningful to ask whether A is also an element of B.}
	      }
	\item{
	      \textbf{Axiom (Empty set).}
	      \emph{There exists a set \(\phi\), known as the empty set,which contains no elements, i.e., for every object \(x\) we have \(x \notin \phi\).}}
	\item{
	      \textbf{Axiom (Replacement).}
	      \emph{Let \(A\) be a set. For any object \(x \in A\), and any object \(y\), suppose we have a statement \(P(x, y)\) pertaining to \(x\) and \(y\),
		      such that for each \(x \in A\) there is at most one \(y\) for which \(P(x, y)\) is true.
		      Then there exsists a set \(\{y: P(x, y)\ is\ true\ for\ some\ x \in A\}\), such that for any object \(z\),
	      }
	      \[\begin{aligned}
			      z \in \{ y:{} & P(x, y)\ is\ true\ for\ some\ x\ \in A \}        \\
			                    & \Leftrightarrow P\ is\ true\ for\ some\ x \in A.
		      \end{aligned}\]
	      }
\end{enumerate}

\emph{(De Morgan laws) We have
	\(X \backslash (A \cup B) = (X \backslash A) \cap (X \backslash B)\)} and \(X \backslash (A \cap B) = (X \backslash A) \cup (X \backslash B)\).

And we use \emph{cardinality} to describe the counts of the elements in a set.


\textbf{Equality of functions.} \emph{Two functions f, g with the same domain and rage are said to be equal, if and only if for every \(x \in X\)}, \(f(x) = g(x)\).

Let's look at the function \(f: \phi \rightarrow X\), from the empty set to an arbitrary set \(X\).
We can say that for every \(X\), there's only one empty function.

\textbf{Lemma (Composition is associative).} \emph{
	Let \(f: Z \rightarrow W\), \(g: Y \rightarrow Z\), \(h: X \rightarrow Y\) be funcitons.
	Then \(f \circ (g \circ h)=(f \circ g) \circ h\).
}

A function is \emph{one-to-one}(or \emph{injective}) if \[
	f(x) = f(x') \Rightarrow x = x'
\]

A function is \emph{onto}(or \emph{surjective}) if \[
	\forall y \in Y, \exists x \in X such that f(x) = y
\]

If \(f\) is both onto and one-to-one, then it's \emph{bijective}.

If \(f\) is bijective, then for every \(y \in Y\), there is exactly one \(x\) such that \(f(x) = y\).
This value of \(x\) is denoted as \(f^{-1}(y)\).
Thus \(f^{-1}\) is a function from \(Y\) to \(X\), the reverse of \(f\).

\end{document}
